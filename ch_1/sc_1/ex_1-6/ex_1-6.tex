\documentclass{article}
\usepackage[utf8]{inputenc}
\usepackage{indentfirst}

\title{SICP Exercise 1-6}
\author{Samuel Lair}
\date{January 19, 2020}

\begin{document}

\maketitle

\section{Solution}

This exercise is equivalent to asking what will happen when Scheme interprets the following program:

\begin{verbatim}
        (define (average x y)
            (/ (+ x y) 2))
        (define (improve guess x)
            (average guess (/ x guess))
        (define (square x) (* x x))
        (define (abs x)
            (cond ((< x 0) (- x))
                  (else x)))
        (define (good-enough? guess x)
            (< (abs (- (square guess) x)) 0.001)
        (define (new-if predicate then-clause else-clause)
            (cond (predicate then-clause)
                (else else-clause)))
        (define (sqrt-iter guess x)
            (new-if (good-enough? guess x)
                guess
                (sqrt-iter (improve guess x)
                           x)))
        (sqrt-iter 1 2)
\end{verbatim}

Scheme uses applicative-order evaluation.  The built-in if is a special form where only the branch chosen by the value of the predicate is evaluated.  However, new-if is an operator.  Therefore, both of the operands will evaluated regardless of the value of the predicate.

This poses a problem because sqrt-iter is a recursive operator.  Even when the base case is satisfied (i.e. the guess is good-enough), the Scheme interpeter will still evaluate yet another sqrt-iter combination.  Therefore, the Scheme interpreter is stuck in an infinite loop when it evaluates (sqrt-iter 1 2). 

However, this problem is solved when if is used instead of new-if.  In this version of the program, when the guess is good-enough, guess is evaluated without evaluating the recursive combination.
\end{document}
