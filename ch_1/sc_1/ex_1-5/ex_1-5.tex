\documentclass{article}
\usepackage[utf8]{inputenc}
\usepackage{indentfirst}

\title{SICP Exercise 1-5}
\author{Samuel Lair}
\date{January 19, 2020}

\begin{document}

\maketitle

\section{Introduction}

Ben Bitdiddle executes the following program:

\begin{verbatim}
        (define (p) (p))
        (define (test x y)
            (if (= x 0)
               0
              y))
        (test 0 (p))
\end{verbatim}

Describe what happens under applicative-order and normal-order evaluation.



\section{Applicative-Order Evaluation}

Under applicative-order evaluation, operators and operands are evaluated before applying the resulting procedures.  This poses a problem when evaluating (test 0 (p)).  The definition of (p) is recursive with no terminating base case.  Therefore, an applicative-order interpreter will get stuck in an infinite loop when it tries to evaluate the (p) in (test 0 (p)).

The Scheme interpreter uses applicative-order evaluation so this is what actually happens when this program is evaluated.

\section{Normal-Order Evaluation}
Under normal-order evaluation, operands are not evaluated until their values are needed.  Therefore, (test 0 (p)) is expanded to:

\begin{verbatim}
        (if (= 0 0)
            0
            (p)))
\end{verbatim}

Since 0 equals 0, a normal-order interpreter will return 0 without evaluating (p).

\end{document}
